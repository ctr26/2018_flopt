\documentclass{letter}
\usepackage{hyperref}
\signature{Craig Russell}
\address{}
\begin{document}

\begin{letter}{}
\opening{Dear Editor}

We submit the attached document for consideration as a letter in Scientific Reports. The title of our paper is ``Frame Localisation Optical Projection Tomography''

In this paper we present new a tomographic reconstruction algorithm which is applied to Optical Projection Tomography (OPT). Volumetric imaging approaches in microscopy are rapidly becoming the standard for interacting with more relevant 3D biological tissues, as such, OPT is seeing a growing interest and adoption in the laboratory. However, OPT is often plagued by reconstruction artefacts due to mechanical instabilities that are difficult to address. This paper presents an algorithm that is robust to mechanical instability through the use of multiple (5+) tracked fiducial beads whereby the sample pose is recovered and image rays are back-projected at all orientations. Our approach shows an improvement when compared to the standard Radon transform, particularly when systematic representative drifts (angular and spatial) are introduced.

%
% present a new method for reconstructing 3D image volumes from 2D image projections using
%
%
% We present a tomographic reconstruction algorithm, which is applied to Optical
% Projection Tomography (OPT) [1] images, that is robust to mechanical jitter and systematic
% angular and spatial drift. OPT relies on precise mechanical rotation and is less mechanically
% stable than large scale CT scanning systems, leading to reconstruction artefacts. The algorithm
% uses multiple (5+) tracked fiducial beads to recover the sample pose and the image rays are then
% back-projected at each orientation. The quality of the image reconstruction using the proposed
% algorithm shows an improvement when compared to the Radon transform. When adding a
% systematic spatial and angular mechanical drift, the reconstruction using the proposed algorithm
% shows a significant improvement over the Radon transform.

%
% In this paper we compare methods for controlling the registration of virtual light-sheets to the imaging plane in light sheet microscopy. We demonstrate experimentally that light sheet registration using projective transforms (4-point registration) produces substantially better image quality, due to lower background fluorescence, than the quality achieved with 3-point registration that is frequently implemented by system developers. We provide a clear exposition of the mathematics of projective transforms in this role, which we believe will be helpful to system developers and of general interest to the light sheet microscopy community. We also provide open source Labview and Matlab softwares to implement the methods described in the paper.

\closing{Yours faithfully,}

% \ps
%
% P.S. You can find the full text of GFDL license at
% \url{http://www.gnu.org/copyleft/fdl.html}.
%
% \encl{Copyright permission form}

\end{letter}
\end{document}
